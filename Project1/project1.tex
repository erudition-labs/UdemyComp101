\documentclass[11]{article}
\usepackage[margin=1in]{geometry}
\usepackage{amsfonts,amsmath,amssymb}
\usepackage{fancyhdr}
\usepackage{graphicx}
\usepackage{float}
\usepackage{transparent}
\usepackage{eso-pic}
\usepackage[colorlinks,linkcolor={blue}]{hyperref}



\usepackage{listings}
\usepackage{color}


\definecolor{dkgreen}{rgb}{0,0.6,0}
\definecolor{gray}{rgb}{0.5,0.5,0.5}
\definecolor{mauve}{rgb}{0.58,0,0.82}

\lstset{frame=tb,
  language=Java,
  aboveskip=3mm,
  belowskip=3mm,
  showstringspaces=false,
  columns=flexible,
  basicstyle={\small\ttfamily},
  numbers=none,
  numberstyle=\tiny\color{gray},
  keywordstyle=\color{blue},
  commentstyle=\color{dkgreen},
  stringstyle=\color{mauve},
  breaklines=true,
  breakatwhitespace=true,
  tabsize=3
}



\newcommand\BackgroundPic{%
\put(0,0){%
\parbox[b][\paperheight]{\paperwidth}{%
\vfill
\centering
{\transparent{0.3} \includegraphics[width=\paperwidth,height=\paperheight,%
keepaspectratio]{background.jpg}}%
\vfill
}}}

\AddToShipoutPicture*{\BackgroundPic}

\pagestyle{fancy}
\fancyhead{}
\fancyfoot{}
\fancyhead[L]{\slshape \MakeUppercase{Notes}}
\fancyfoot[C]{\thepage}
%\renewcommand{\headrulewidth}{0pt}
\renewcommand{\footrulewidth}{0pt}

\parindent 0ex
\renewcommand{\baselinestretch}{1.5}

\begin{document}
\begin{titlepage}
\begin{center}
\vspace{1cm}
\Large{\textbf{Computer Science 101: Introduction to Java and Algorithms}}\\
\vfill
\line(1,0){400}\\
\huge{\textbf{Project 1: Game Of Life}}\\
\line(1,0){400}\\
\vfill
Erudition Labs\\
Computer Science 101: Introduction to Java and Algorithms\\
\today\\
\end{center}
\end{titlepage}

\tableofcontents
\thispagestyle{empty}
\clearpage
\setcounter{page}{1}
\section{John Conway`s Game of Life}
Most of this introduction comes from \url{<https://en.wikipedia.org/wiki/Conway\%27s_Game_of_Life>} If you would like to just read about it there. John Conway was a mathematician who was trying to create a simulation that would fit computer scientist, John von Neumann`s definition of what life is. Von Neumann was trying to do this by engineering a solution by making electromagnetic components randomly float in liquids or a gas. Another mathematician by the name of Stanislaw Ulam invented ``cellular automata`` to simulate what Von Neumann was trying to do. Finally, John Conway created a set of complex mathematical rules that would yield a configuration that would fit Von Neumann`s definition of life. That being said, Conway was more interested in trying to be able to complex and unpredictable cellular automatons that could go on forever or die out depending on the initial configuration. It`s and interesting way to see patterns evolve and show the emergence of organisms as well as their self-organization with Conway's deterministic rules.

\subsection{Cellular Automata}
Cellular automaton consists of a grid of cells that have a finite number of states and where rules are applied to all of the cells. It is quite commonly using in simulations in the sciences like physics, chemistry etc. We will be talking about how we will do this in a bit.

\subsection{The Rules}
Many conclusions have been drawn from Conway`s experiments. His mathematical rules can be summed up in words as the following.

We assume that we have an infinite 2d, orthogonal grid.

\begin{enumerate}
  \item Any live cell with fewer than two live neighbours dies, as if by underpopulation.
  \item Any live cell with more than three live neighbours dies, as if by overpopulation.
  \item Any dead cell with exactly three live neighbours becomes a live cell, as if by reproduction.
  \item Any live cell with two or three live neighbours lives on to the next generation.
\end{enumerate}

As it turns out, these rules, when applied to cellular automaton, will create complex patterns and form groups of cells (organisms) that do things like oscillating pulsation and even move.

\section{Our Project} 
We are going to implement this simulation using most of the concepts we have learned so far like variables, arrays, loops, if-statements and methods. We are going to translate the above descriptions and rules into working code.

\subsection{Modeling}
\subsubsection{2D infinite Grid and Cell States}
Lets think about how we can do this. The rules require an infinite 2d grid. Obviously, we cannot do the infinite part since we do not have infinite memory. However, we can still operate under the assumption that the grid is infinite.\\

We can use a two-dimensional array to represent our infinite 2d grid. We will need it to be fixed size, so we are going to have to break the rules a bit. When a cell is about to move past the limits of our 2d array, what are we going to do? Well one option is to wrap it around to the other side. However, I get a less infinite feeling form this. Another option, and the one we will most likely do, is to just kill the cell once is surpasses the confines of our grid.\\

So we have a 2d array to represent out 2d grid, and each element of our 2d array will be a ``cell``. In the rules, there are only two states that a cell can be in, alive or dead. There are a number of ways that we could represent those states. Perhaps we could use boolean values. If we create a 2d array of type boolean, then we could let ``true`` be ``alive`` and ``false`` be dead. This is probably the simplest way, but it is also limiting to two states. As we go through the course, we are going to be re-factoring this code as well as adding to it. I plan on adding to the rules and the sates. So to give us more options, we are going to use an integer array. This way, we can use ``0`` as dead and ``1`` as alive.\\

So again, we are going to represent our infinite 2d grid as a 2d integer array where once a cell leaves the confines of our array, we will simply assume that it will go on forever without really affecting our immediate system, so we will just kill that cell.

\subsubsection{Rules}
Now how can we apply the rules that govern that states of the cells. Well we could use a switch statement or if-statements to check each of the rules. We will loop over each element in our 2d array. Keep in mind that each element is a cell that is either dead or alive. We will use the current index to calculate where the neighbors of that cell is at. Then we will index into the array where the neighbor positions are to check their states. Finally, we will use their states to decide if the current cell should live or die. Once we have iterated over the entire array, that will be considered one generation.

\subsubsection{Visualization}
To Keep things simple, for now, we will simply be printing all of the live cells in the standard output, aka terminal. In other words, the simulation will be text based for now.
\end{document}