\documentclass[11]{article}
\usepackage[margin=1in]{geometry}
\usepackage{amsfonts,amsmath,amssymb}
\usepackage{fancyhdr}
\usepackage{graphicx}
\usepackage{float}
\usepackage{transparent}
\usepackage{eso-pic}
\usepackage{hyperref}


\usepackage{listings}
\usepackage{color}


\definecolor{dkgreen}{rgb}{0,0.6,0}
\definecolor{gray}{rgb}{0.5,0.5,0.5}
\definecolor{mauve}{rgb}{0.58,0,0.82}

\lstset{frame=tb,
  language=Java,
  aboveskip=3mm,
  belowskip=3mm,
  showstringspaces=false,
  columns=flexible,
  basicstyle={\small\ttfamily},
  numbers=none,
  numberstyle=\tiny\color{gray},
  keywordstyle=\color{blue},
  commentstyle=\color{dkgreen},
  stringstyle=\color{mauve},
  breaklines=true,
  breakatwhitespace=true,
  tabsize=3
}



\newcommand\BackgroundPic{%
\put(0,0){%
\parbox[b][\paperheight]{\paperwidth}{%
\vfill
\centering
{\transparent{0.3} \includegraphics[width=\paperwidth,height=\paperheight,%
keepaspectratio]{background.jpg}}%
\vfill
}}}

\AddToShipoutPicture*{\BackgroundPic}

\pagestyle{fancy}
\fancyhead{}
\fancyfoot{}
\fancyhead[L]{\slshape \MakeUppercase{Notes}}
\fancyfoot[C]{\thepage}
%\renewcommand{\headrulewidth}{0pt}
\renewcommand{\footrulewidth}{0pt}

\parindent 0ex
\renewcommand{\baselinestretch}{1.5}

\begin{document}
\begin{titlepage}
\begin{center}
\vspace{1cm}
\Large{\textbf{Computer Science 101: Introduction to Java and Algorithms}}\\
\vfill
\line(1,0){400}\\
\huge{\textbf{Section 2: Control Statements - Selections}}\\
\line(1,0){400}\\
\vfill
Erudition Labs\\
Computer Science 101: Introduction to Java and Algorithms\\
\today\\
\end{center}
\end{titlepage}

\tableofcontents
\thispagestyle{empty}
\clearpage
\setcounter{page}{1}

\section{Relational and Logical Operators (Video Series Lecture 7)}
In this part of the series, we introduce the relational operators.  Interestingly enough, they are all the same operators that exist in basic mathematics. However, this would be our first experience with logic. By that I mean that it will be helpful here to start thinking of things in terms of true and false. For example, $1<2$ is a true statement. Whereas $5>1$ is a false statement. That`s all there really is to say about it.
\subsection{The Relational Operators}
\begin{center}
  \begin{tabular}{ | c | c | c | c | c |}
    \hline
    Math Symbol & Programming Symbol & Definition & Example & Value \\ \hline
    $>$ & $>$ & is greater than & $5$ $>$ $4$ & true \\ \hline
    $\geq$ & $>=$ & is greater than or equal to & $5$ $>=$ $5$ & true \\ \hline
    $<$ & $<$ & is less than & $4$ $<$ $10$ & false \\ \hline
    $\leq$ & $<=$ & is less than or equal to & $5$ $<=$ $10$ & true \\ \hline
    $=$ & $==$ & is equal to & $8$ $==$  $4$ & false \\ \hline
    $\neq$ & $!=$ & is not equal to & $8$ $!=$  $4$ & true \\ 
    \hline
  \end{tabular}
\end{center}

Take note that in programming, we combine symbols such as $<=$ instead of $\leq$, why? Well simply because, do you see $\leq$ on your keyboard? If you do, then I have no idea what keyboard you are using. \\

A more important one to note is the 	`equal to` operator. In mathematics we use $=$ as both assignment and to express an equality relation. In fact, we generally use them as one in the same. However, a computer needs to make a distinction due to additional things the compiler must do. In programming, we use $=$ as assignment. When we say 
\begin{lstlisting}
int a = 5;
\end{lstlisting}
what we are actually saying is, hey compiler, create a variable in memory with enough space to store an integer and store the value $5$ there. Now take the case, 
\begin{lstlisting}
int a = 5;
int b =0;
a = b;
\end{lstlisting}

Here we tell the compiler to again create the variable $a$ and store the value $5$ and then create the variable $b$ and store the value $0$. Then we are assigning the value that is stored in the variable $a$ into the variable $b$. That is, we are copying the value stored in variable $a$ into $b$. So when we try to ask the program, ``is $a$ equal to $b$`` we cannot use the $=$ symbol. If we did, the computer would think that we are trying to assign something. Instead we use the symbols $==$ to ask if something is equal. This will make more sense when we try to use these operators.\\

\subsection{Logical Operators}
The logical operators are directly rooted into propositional logic. What`s great is that we use these everyday, often without even realizing it.

\begin{center}
  \begin{tabular}{ | c | c | c | c | c |}
    \hline
    Logic & Programming Symbol & Definition & Example & Value \\ \hline
    $\& (Conjunction)$ & $\&\&$ & AND & ($5==5$) $\&\&$ ($5!=10$) & true \\ \hline
    $V (Disjunction)$ & $||$ & OR & ($3==3$) $||$ ($5==3$) & true \\ \hline
    $\lnot (Negation)$ & $!$ & NOT &  $!(5 != 5)$ & true \\ 
    \hline
  \end{tabular}
\end{center}

In programming (and logic) you can combine these logical operations to evaluate to some boolean value.\\

\subsubsection{AND}
In propositional logic, AND is a conjunction of two statements. We as humans have learned to communicate in this way as well, so it is often helpful to say it as a sentence to yourself and see if it makes sense. For example, ``Fire is hot and fire burns wood``. The two statements, ``Fire is hot`` and ``fire burns wood`` are both true about fire. We know this to be true because of our senses. since the two statements are true, then the conjunction of the two statements must also be true.\\

Let us look at the truth table for the conjunction AND (I will be using the programming symbols for the logical operators),
\begin{center}
  \begin{tabular}{ | c | c | c |}
    \hline
    A & B & A $\&\&$ B \\ \hline
    false & false & false \\ \hline
    false & true & false \\ \hline
    true & false & false \\ \hline
    true & true & true \\
    \hline
  \end{tabular}
\end{center}

As you can see, the only time the AND operator is true is when both statements are true. If you really think about it, this makes sense when you speak. The next table will just be in plain English.

\begin{center}
  \begin{tabular}{ | c | c | c | c | p{4.5cm} |}
    \hline
		Statement A & Statement B & A $\&\&$ B & value & Explanation \\ \hline
		Fire is cold & Fire is wet & Fire is cold and Fire is wet & false & We know fire isn`t cold or wet so it also can`t be both cold and wet \\  \hline
		
		Fire is cold & Fire is hot & Fire is cold and Fire is hot & false & Fire is hot, but we know it isn`t cold, so the sentence really makes no sense \\ \hline
		
		Fire is hot & Fire is cold & Fire is hot and Fire is cold & false & moving the true part of the sentence to the other side doesn`t change the fact that fire isn`t both hot and cold \\ \hline
		
		Fire is hot & Fire burns wood & fire is hot and fire burns wood & true & we know that fire is hot and that fire burns wood, so fire must be hot and also it must burn wood, which makes sense. \\
    \hline
  \end{tabular}
\end{center}

The next table, I will use conjunctions of relation statements, similar to what we would find in programming. In this table we will use a variable, $i$, let $i$ $=$ $5$

\begin{center}
  \begin{tabular}{ | c | c | c | c |}
    \hline
    A & B & A $\&\&$ B & Value\\ \hline
    ($i$ $==$ $3$) & ($i$ $>=$ $10$) & ($i$ $==$ $3$) $\&\&$ ($i$ $>=$ $10$) & false \\ \hline
    ($i$ $<$ $3$) & ($i$ $<=$ $5$) & ($i$ $<$ $3$) $\&\&$ ($i$ $<=$ $5$) & false \\ \hline
    ($i$ $==$ $5$) & ($i$ $!=$ $5$) &  ($i$ $==$ $5$) $\&\&$ ($i$ $!=$ $5$)& false \\ \hline
    ($i$ $>=$ $5$) & ($i$ $<=$ $5$) & ($i$ $>=$ $5$) $\&\&$ ($i$ $<=$ $5$) & true \\
    \hline
  \end{tabular}
\end{center}

Note that these two tables both correspond to the AND truth table. Now I would like to do a more complicated example with conjunctions of relational statements. Let $i$ $=$ $5$ again. Now let us evaluate\\ ((($i$ $!=$ $5$) $\&\&$ ($i$ $==$ $10$)) $\&\&$ $i$ $<=$ $5$)\\

First identify the order, we must do the inner parenthesis first.\\
(($i$ $!=$ $5$) $\&\&$ ($i$ $==$ $10$))\\
Now we need to identify the statements\\
($i$ $!=$ $5$) and ($i$ $==$ $10$) \\
Now evaluate these statements \\
($i$ $!=$ $5$) is false\\
 ($i$ $==$ $10$) is false \\
 So, (($i$ $!=$ $5$) $\&\&$ ($i$ $==$ $10$)) is equivalent to ((false) $\&\&$ (false)).\\
 We can consult the Truth table to see that (($i$ $!=$ $5$) $\&\&$ ($i$ $==$ $10$)) is false\\
 
 Now we have a conjunction of (false $\&\&$ ($i$ $<=$ $5$))\\
 we know that ($i$ $<=$ $5$) is true \\
 So, we have reduced ((($i$ $!=$ $5$) $\&\&$ ($i$ $==$ $10$)) $\&\&$ $i$ $<=$ $5$) to (false $\&\&$ true)\\
 We can again consult the truth table to see that (false $\&\&$ true) is false.\\
 
 Therefore, ((($i$ $!=$ $5$) $\&\&$ ($i$ $==$ $10$)) $\&\&$ $i$ $<=$ $5$) evaluates to false.
 
 \subsubsection{OR}
 In propositional logic, OR is the disjunction of two statements. For example, ``It is sunny`` or ``it is cloudy``. It is one or the other. The truth table looks like this:\\
 
 \begin{center}
  \begin{tabular}{ | c | c | c |}
    \hline
    A & B & A $||$ B \\ \hline
    false & false & false \\ \hline
    false & true & true \\ \hline
    true & false & true \\ \hline
    true & true & true \\
    \hline
  \end{tabular}
\end{center}

As you can see, the only time that the whole disjunction is false is when both statements are false. This makes sense when we talk as well. When using OR we are speaking in terms of one or the other or both. When it comes to OR, we only need one statement to be true to make the whole disjunction true. Here is an English truth table.

\begin{center}
  \begin{tabular}{ | c | c | c | c | p{4.5cm} |}
    \hline
		Statement A & Statement B & A $||$ B & value & Explanation \\ \hline
		Fire is cold & Fire is wet & Fire is cold or Fire is wet & false & We know fire isn`t cold or wet so it is neither. \\  \hline
		
		Fire is cold & Fire is hot & Fire is cold or Fire is hot & true & Fire is hot, but we know it isn`t cold, but since fire is hot, we know it is at least one of those two options \\ \hline
		
		Fire is hot & Fire is cold & Fire is hot or Fire is cold & true & moving the true part of the sentence to the other side doesn`t change the fact that fire is hot even if it isn`t cold \\ \hline
		
		Fire is hot & Fire burns wood & fire is hot or fire burns wood & true & we know that fire is hot and that fire burns wood, so fire must be hot and also it must burn wood, so take your pick. \\
    \hline
  \end{tabular}
\end{center}

Now again, but using relational statements that you would see in programming. Let $i$ $=$ $5$
\begin{center}
  \begin{tabular}{ | c | c | c | c |}
    \hline
    A & B & A $||$ B & Value\\ \hline
    ($i$ $==$ $3$) & ($i$ $>=$ $10$) & ($i$ $==$ $3$) $||$ ($i$ $>=$ $10$) & false \\ \hline
    ($i$ $<$ $3$) & ($i$ $<=$ $5$) & ($i$ $<$ $3$) $||$ ($i$ $<=$ $5$) & true \\ \hline
    ($i$ $==$ $5$) & ($i$ $!=$ $5$) &  ($i$ $==$ $5$) $||$ ($i$ $!=$ $5$)& true \\ \hline
    ($i$ $>=$ $5$) & ($i$ $<=$ $5$) & ($i$ $>=$ $5$) $||$ ($i$ $<=$ $5$) & true \\
    \hline
  \end{tabular}
\end{center}

Lets use the same statement from the above subsection, but swap out the AND`s with OR`s and see what happens to the value. Let us evaluate: \\
((($i$ $!=$ $5$) $||$ ($i$ $==$ $10$)) $||$ $i$ $<=$ $5$)\\

First identify the order, we must do the inner parenthesis first.\\
(($i$ $!=$ $5$) $||$ ($i$ $==$ $10$))\\
Now we need to identify the statements\\
($i$ $!=$ $5$) or ($i$ $==$ $10$) \\
Now evaluate these statements \\
($i$ $!=$ $5$) is false\\
 ($i$ $==$ $10$) is false \\
 So, (($i$ $!=$ $5$) $||$ ($i$ $==$ $10$)) is equivalent to ((false) $||$ (false)).\\
 We can consult the Truth table to see that (($i$ $!=$ $5$) $||$ ($i$ $==$ $10$)) is false\\
 
 Now we have a disjunction of (false $||$ ($i$ $<=$ $5$))\\
 we know that ($i$ $<=$ $5$) is true \\
 So, we have reduced ((($i$ $!=$ $5$) $||$ ($i$ $==$ $10$)) $||$ $i$ $<=$ $5$) to (false $||$ true)\\
 We can again consult the truth table to see that (false $||$ true) is true.\\
 
 Therefore, ((($i$ $!=$ $5$) $||$ ($i$ $==$ $10$)) $||$ $i$ $<=$ $5$) evaluates to true.
 
 \subsubsection{NOT}
 In propositional logic, NOT is the equivalent to the negation. Another way to think of this is ``the opposite of``. Obviously not true is false and not false is true, so the truth table looks like:\\
 
  \begin{center}
  \begin{tabular}{ | c | c |}
    \hline
    A & !A  \\ \hline
    false & true\\ \hline
    true & false\\ 
    \hline
  \end{tabular}
\end{center}

It`s sort of like saying a statement and then pausing at the end and saying ``just kidding!``. Like this, ``Fire is cold...JUST KIDDING``, we just took a false statement and made it true.

\begin{center}
  \begin{tabular}{ | c | c | c | p{4.5cm} |}
    \hline
		Statement A  & !A & value & Explanation \\ \hline
		Fire is cold & Fire is cold...NOT & true & We know fire is not cold, but we want the opposite value\\  \hline
		
		Fire is hot & Fire is hot...NOT & false & We know fire is hot, but we want the opposite value\\ \hline		
  \end{tabular}
\end{center}

Let`s see if we can make our example using the OR`s false by using NOT\\
((($i$ $!=$ $5$) $||$ ($i$ $==$ $10$)) $||$ $i$ $<=$ $5$)\\

First identify the order, we must do the inner parenthesis first.\\
(($i$ $!=$ $5$) $||$ ($i$ $==$ $10$))\\
Now we need to identify the statements\\
($i$ $!=$ $5$) or ($i$ $==$ $10$) \\
Now evaluate these statements \\
($i$ $!=$ $5$) is false\\
 ($i$ $==$ $10$) is false \\
 So, (($i$ $!=$ $5$) $||$ ($i$ $==$ $10$)) is equivalent to ((false) $||$ (false)).\\
 We can consult the Truth table to see that (($i$ $!=$ $5$) $||$ ($i$ $==$ $10$)) is false\\
 
 Now we have a disjunction of (false $||$ ($i$ $<=$ $5$))\\
 we know that ($i$ $<=$ $5$) is true \\
 So, we have reduced ((($i$ $!=$ $5$) $||$ ($i$ $==$ $10$)) $||$ $i$ $<=$ $5$) to (false $||$ true)\\
 We can again consult the truth table to see that (false $||$ true) is true.\\
 
 Therefore, ((($i$ $!=$ $5$) $||$ ($i$ $==$ $10$)) $||$ $i$ $<=$ $5$) evaluates to true.\\
 
 To make the whole disjunction false, we need to make the second statement false. So, what if we change the expression to \\
 ((($i$ $!=$ $5$) $||$ ($i$ $==$ $10$)) $||$ !($i$ $<=$ $5$))\\
 
 First identify the order, we must do the inner parenthesis first.\\
(($i$ $!=$ $5$) $||$ ($i$ $==$ $10$))\\
Now we need to identify the statements\\
($i$ $!=$ $5$) or ($i$ $==$ $10$) \\
Now evaluate these statements \\
($i$ $!=$ $5$) is false\\
 ($i$ $==$ $10$) is false \\
 So, (($i$ $!=$ $5$) $||$ ($i$ $==$ $10$)) is equivalent to ((false) $||$ (false)).\\
 We can consult the Truth table to see that (($i$ $!=$ $5$) $||$ ($i$ $==$ $10$)) is false\\
 
 Now we have a disjunction of (false $||$ !($i$ $<=$ $5$))\\
 we know that ($i$ $<=$ $5$) is true, so   !($i$ $<=$ $5$) must be false\\
 So, we have reduced ((($i$ $!=$ $5$) $||$ ($i$ $==$ $10$)) $||$ !($i$ $<=$ $5$)) to (false $||$ false)\\
 
 We can consult the OR truth table to see that false $||$ false is false.

\section{True and False (Video Series Lecture 8 and 9)}
In this section I would like to just do more examples.
\section{if Statement (Video Series Lecture 10 and 11)}
We have talked about all of these relational and logical operators and their evaluations all for this moment (and others, but this first moment). These are your tools for evaluating conditions. We combine these conditions with statements. The first being the if statement.\\
\subsection{if}
\begin{lstlisting}
if(CONDITION_IS_TRUE) {
    // DO THIS STUFF
}
\end{lstlisting}

You would use the if-statement whenever you only wanted to execute some code when a condition is met.The if statement simply says, ``if this condition is true, then execute the next block of code.`` You could also imply that if the condition is true, then execute the next block of code, but if the condition is false, don`t execute the next block of code.\\

If statements are pretty crucial to programming. For example, we all have little bombs inside our cars. The air bag in your car inflates with gas when you slow down real quick, in fact, so intensely that you would go from 55 mph to 0mph in a split second. Well that explosive in your steering wheel is regulated by an accelerometer that is constantly getting your deceleration and probably using an if statement to check if your current deceleration is greater than or equal to the deceleration that you would see in a car crash. If it is, then detonate the explosive so that the gas from the explosion can expand and fill the air bag. If statements literally save your life in cases of terrible car accidents.\\

Or lets say you are creating a banking system and you only want to allow a user to take out money if they have enough funds to do so. You would need to get the current amount of money in the bank and use an if statement to check if the bank amount is greater than the amount of money the user wishes to withdraw. If it is, then execute the code to withdraw the requested amount.\\

There are often numerous variables and conditions that you may want in place before some piece of code is executed. This is the use case of the if statement.\\

A note of caution however. We have been doing some pretty long and complicated conditions for practice. This is generally a very bad idea when programming. You want your conditions to be as simple and easy to read as possible. If they get too long or complicated, then you need to rethink your design. That being said, sometimes you have no choice but to use long, ugly conditions.
\subsection{Full Code Examples}
\begin{lstlisting}
public class HelloWorld{

     public static void main(String []args){
        int menuOption = 2;
         
        if(menuOption == 1) {
            System.out.println("Option 1: Enter Game");
        }
        
        if(menuOption == 2) {
            System.out.println("Option 2: You are Playing");
        }
        
        if(menuOption == 3) {
            System.out.println("Option 1: Enter");
        }        
     }
}
\end{lstlisting}

\section{if else and if else if Statements (Video Series Lecture 12 and 13)}
\subsection{if else}
Remember in the above section when i described the if statement as follows: ``if the condition is true, then execute the next block of code, but if the condition is false, don`t execute the next block of code.`` Well what if we want to do something if the condition fails. What if we could reword it as follows: ``if the condition is true, then execute the next block of code, but if the condition is false, don`t execute the next block of code, but instead execute this other block of code``. What if we have a bunch of if statements and none of them pass, should we just do nothing? Or can we do something in the case that all conditions are evaluated to false?\\

In comes the if-else statement. It is literally translated as, if the condition is true, execute this code block, otherwise, execute this other code block.

\begin{lstlisting}
if(SOME_CONDITION) {
    // DO THIS STUFF
} else {
    //IF NONE OF THE CONDITIONS ARE TRUE
    //THEN DO THIS STUFF
}
\end{lstlisting}

If we go back to the banking system idea where we only want to allow a user to take out money that they actually have in the bank, we can use an if statement to check if the funds are available. If they are, then execute give them their money, otherwise, notify them that they do not have the funds.
\begin{lstlisting}
if(accountBalance > withdrawRequestAmount) {
    // Give them their money
} else {
	// HEY, YOU HAVE NO MONEY TO TAKE OUT
}
\end{lstlisting}

\subsection{if-else if-else}
Let`s continue with our really dumb bank example. Check out this code:

\begin{lstlisting}
public class Bank {

     public static void main(String []args){
        int accountAmount = 1000;
        int withdrawRequestAmount = 1000;
        
        if(accountAmount > withdrawRequestAmount) {
            System.out.println("Heres your money");
        } else {
            System.out.println("GET A JOB, YOU NEED MONEY");
        }
        
        if(accountAmount == withdrawRequestAmount) {
            System.out.println("Heres your money");
        } else {
            System.out.println("GET A JOB, YOU NEED MONEY");
        }
     }
}
\end{lstlisting}

The first two lines of our main method, we just initialize accountAmount and withdrawRequestAmount to $\$1000$. That`s how much I have at the bank and just so happens to be how much I want to take out of the bank. The first if statement will print ``GET A JOB, YOU NEED MONEY`` because accountAmount is not greater than withdrawRequestAmount. Since the if statement evaluates to false, my else block gets executed. The second if statement checks if they are equal and in this case they are, so it prints ``Here`s your money``.  \\

Now wait a second. First the bank to tells me to get a job because I don`t have enough money, and then it gives me my money? What is going on? Since these are all separate if statements, each one gets checked every single time the main method runs. Also note that I have put the same else block for all of my if statements.\\

But what if I just want to check if statements and stop checking them once one is true. Or what if I want to use the same else statement for all the if statements, but I don`t want to write it over and over. And how do I stop my program from printing those two outputs when I only want one.\\

In comes the if-else if-else statement. With these statements, the program will go through the if statements until one of them is true, and then it will skip the rest. If none of them are true, then it will execute the else block, which you only need to write once.\\

We can refactor the code to take advantage of if-else if-else statements.
\begin{lstlisting}
public class Bank {

     public static void main(String []args){
        int accountAmount = 1000;
        int withdrawRequestAmount = 1000;
        
        if(accountAmount > withdrawRequestAmount) {
            System.out.println("Heres your money");
        } else if(accountAmount == withdrawRequestAmount) {
            System.out.println("Heres your money, you have none left now");
        } else {
            System.out.println("GET A JOB, YOU NEED MONEY");
        }
     }
}
\end{lstlisting}
We have the same setup as before, accountAmount and withdrawRequestAmount are initialized to $\$1000$.
The if statement evaluates to false, so it goes to the else if statement, which evaluates to true, so it prints out ``Heres your money, you have none left now`` and then skips the else. Note that we can have as many else if statements as we want. Once the program finds one that is true, it will skip the rest.
\section{switch Statement (Video Series Lecture 14 and 15)}
\end{document}