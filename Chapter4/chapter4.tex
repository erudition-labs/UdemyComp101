\documentclass[11]{article}
\usepackage[margin=1in]{geometry}
\usepackage{amsfonts,amsmath,amssymb}
\usepackage{fancyhdr}
\usepackage{graphicx}
\usepackage{float}
\usepackage{transparent}
\usepackage{eso-pic}
\usepackage{hyperref}


\usepackage{listings}
\usepackage{color}


\definecolor{dkgreen}{rgb}{0,0.6,0}
\definecolor{gray}{rgb}{0.5,0.5,0.5}
\definecolor{mauve}{rgb}{0.58,0,0.82}

\lstset{frame=tb,
  language=Java,
  aboveskip=3mm,
  belowskip=3mm,
  showstringspaces=false,
  columns=flexible,
  basicstyle={\small\ttfamily},
  numbers=none,
  numberstyle=\tiny\color{gray},
  keywordstyle=\color{blue},
  commentstyle=\color{dkgreen},
  stringstyle=\color{mauve},
  breaklines=true,
  breakatwhitespace=true,
  tabsize=3
}



\newcommand\BackgroundPic{%
\put(0,0){%
\parbox[b][\paperheight]{\paperwidth}{%
\vfill
\centering
{\transparent{0.3} \includegraphics[width=\paperwidth,height=\paperheight,%
keepaspectratio]{background.jpg}}%
\vfill
}}}

\AddToShipoutPicture*{\BackgroundPic}

\pagestyle{fancy}
\fancyhead{}
\fancyfoot{}
\fancyhead[L]{\slshape \MakeUppercase{Notes}}
\fancyfoot[C]{\thepage}
%\renewcommand{\headrulewidth}{0pt}
\renewcommand{\footrulewidth}{0pt}

\parindent 0ex
\renewcommand{\baselinestretch}{1.5}

\begin{document}
\begin{titlepage}
\begin{center}
\vspace{1cm}
\Large{\textbf{Computer Science 101: Introduction to Java and Algorithms}}\\
\vfill
\line(1,0){400}\\
\huge{\textbf{Section 4: Arrays}}\\
\line(1,0){400}\\
\vfill
Erudition Labs\\
Computer Science 101: Introduction to Java and Algorithms\\
\today\\
\end{center}
\end{titlepage}

\tableofcontents
\thispagestyle{empty}
\clearpage
\setcounter{page}{1}

\section{Pre-Chapter}
\subsection{Heap and Stack Memory(Over Simplified)}
\subsection{The ``new`` keyword}
\section{Arrays (Video Series Lecture 20 and 21)}
This section introduces Arrays, which is a data structure that stores collections of elements. What is a data structure? Well a data structure is just a way organizing data by following certain rules when we store it. For now, you don`t need really need to know what a data structures is. If you are taking computer science classes, then there are entire classes dedicated to that topic. All you need to know right now is that an array is used to store a collection of things of the same type.
\subsection{What does an Array Look Like?}
Arrays are contiguous chunks of memory. And they MUST be contiguous due to how we access the data. I will talk about that more later. Usually when we try to visualize and conceptualize what an array is, we draw it like this:\\

\begin{figure}[H]
	\centering
	\includegraphics[scale=0.5]{arrays1.png}
	\caption{Image from Lecture}
\end{figure}

Imagine, if you will, having a bunch of variables of the same type under one name. This is essentially what an array is. So if you need to store a collection of integers, you can declare one array to store them instead of declaring a bunch of different variables. If you look at the table, the top row corresponds to the index position of the array. Notice that we start counting from $0$. The $0$th element is actually the first element in the array. Next, you can see all the boxes. Think of each box as a variable that you don`t have to name that stores a value.
\section{Looping Over Arrays (Video Series Lecture 22 and 23)}
\section{2D Arrays (Video Series Lecture 24 and 25)}
\section{Sorting Arrays: Insertion Sort Algorithm (Video Series Lecture 26 and 27)}
\section{Gotchas with Arrays}



\end{document}